
\chapter{Computer Vision and algorithms}

Computer vision has seen a substancial growth over the last 40 years, as camera technology became mature 
and the analysis of data became cheaper and faster. In this chapter we are going to look at the main algorithms 
that were tested and applied in the project, and give some more depth in the algorithms and mathematichs behind each of them.

\section{Thresholding}

\subsection{Naïve thresholding}

\subsection{Adaptive thresholding}


\section{Bluring}



\section{Optical flow}
Optical flow is a term that covers the apparent motion of objects, spcifically its edges and surfaces 
relative to an observer. The term was first coined for this by James Gibson in the 1940s 
to describe the visual stimulus that animals get from movement\citet{gibson50}.


\subsection{Theory}
The base of optical flow theory is based on the normal flow constraint equation in \eqref{equ:image.constraint}. In equation \eqref{equ:image.constraint}, $I$ is the intensity
at a given point at a given time. It is quite abvious that a normal optical flow problem will be of three dimension as it will include spatial coordinates in addition
to a temporal part. The deltas shown is an indefinite movement in the three dimensions.


\begin{equation}
I(x,y,t) = I(x + \Delta x, y+ \Delta y, t + \Delta t)
\label{equ:image.constraint}
\end{equation}

If the deltas in equation \eqref{equ:image.constraint} is assumed to be small, then the approximation in equation \eqref{equ:image.constraint.taylor} is valid.

\begin{equation}
I(x + \Delta x, y+ \Delta y, t + \Delta t) = I(x,y,t) + \frac{\partial I}{\partial x} \Delta x + \frac{\partial I}{\partial y} \Delta y + \frac{\partial I}{\partial t} \Delta t + \mathcal{O}(\partial^2)
\label{equ:image.constraint.taylor}
\end{equation}

Equation \eqref{equ:image.constraint.taylor} contains a higher order collection term $\mathcal{O}(\partial^2)$. This can be thought of the error in this first order taylor approximation. As
the deltas are assumed small, then the taylor expansion terms with order higher than one is negligible. 

Substituting equation \eqref{equ:image.constraint} into \eqref{equ:image.constraint.taylor} it is obvious that equation \eqref{equ:taylor.exp} must hold. 
By dividing equation \eqref{equ:taylor.exp} with the change of time we get \eqref{equ:taylor.exp.subs}, where $V_x$ and $V_y$ are the spatial velocities in the 
$x$ and $y$ direction.

\begin{align}
\frac{\partial I}{\partial x} \Delta x + \frac{\partial I}{\partial y} \Delta y + \frac{\partial I}{\partial t} \Delta t &= 0 \label{equ:taylor.exp} \\
\frac{\partial I}{\partial x} \frac{\Delta x}{\Delta t} + \frac{\partial I}{\partial y} \frac{\Delta y}{\Delta t} + \frac{\partial I}{\partial t} \frac{\Delta t}{\Delta t} &= \frac{0}{\Delta t} \notag \\
\frac{\partial I}{\partial x} V_x + \frac{\partial I}{\partial y} V_y + \frac{\partial I}{\partial t}  &= 0 \label{equ:taylor.exp.subs}
\end{align}

By defining the intensity spatial derivatives as $I_x$, $I_y$ and $I_t$, we get equation \eqref{equ:intensity.deriv}. Using the dot product on the derivatives 
leads to equation \eqref{equ:optical.flow}, which is the usual way of writing the optical flow problem.

\begin{align}
I_x V_x + I_y V_y &= -I_t \label{equ:intensity.deriv} \\
\nabla I^\top \cdot \vec{V} &= -I_t \label{equ:optical.flow}
\end{align}

Equation \eqref{equ:optical.flow} shows the big problem of optical flow problems. It is a single equation with two unknowns, and is therefore not 
solvable without adding additional constraints to the analysis. The problem is known as the aperture problem, and all of the following algorithms 
introduces some additional condition that inferes constraints to the flow problem.

\subsection{Phase correlation}
As the name of this method implies, this method uses shift and correlation in the phase plane between to frames as a measure of motion. Given that correlation is a global operation,
the method is also image global. The first step is to apply the Fourier transform to both frames, $\textbf{G}_a = \mathcal{F}\{g_a\}$ and $\textbf{G}_b = \mathcal{F}\{g_b\}$. The cross-power spectrum 
can then be calculated with equation \eqref{equ:phase.cross.power}.

\begin{equation}\label{equ:phase.cross.power}
R = \frac{\textbf{G}_a \circ \textbf{G}_b^\star}{|\textbf{G}_a \textbf{G}_b^\star|}
\end{equation}

By inverse transforming $R$, we get the normalized cross-correlation, $r = \mathcal{F}^{-1}\{R\}$, which can be though of as a 
heat map of the movement between the two frames.

There are however some drawbacks with using the phase correlation method, in addition to the global aspect. The global aspect 
inferes smooth and equal movement of all points between the frames. Also, since the Fourier transform is involved. The method has its base 
in the Fourier shift theorem, which implies that the images are assumed to have a circular shift. The case is, however, that the shift 
is more likely to be linear, and this discrepancy gives distortions in the cross-correlation.

\subsection{Lukas-Kanade}
The Lucas-Kanade method is a widely used method for providing the needed constraints to solve \eqref{equ:optical.flow}. This method assumes that 
the motion between to frames is small and more or less constant in the neighborhood of a point, known as the velocity smoothness constraint. 
This means that \eqref{equ:optical.flow} holds for all points within some window
with the point that is being considered. This means that the local velocity vector should satisfy \eqref{equ:lk.local.flow}.

\begin{align}\label{equ:lk.local.flow}
I_x(q_1)V_x + I_y(q_1)V_y	&= -I_t(q_1) \\
I_x(q_2)V_x + I_y(q_2)V_y 	&= -I_t(q_2) \notag \\
&\,\,\, \vdots \notag \\
I_x(q_N)V_x + I_y(q_N)V_y 	&= -I_t(q_N) \notag
\end{align}

Rewriting \eqref{equ:lk.local.flow} on matrix form using the normal $Av = b$ structure, we get the matrices in \eqref{equ:lk.local.flow.mat}.

\begin{equation} \label{equ:lk.local.flow.mat}
A = 
\begin{bmatrix}
	I_x(q_1) 	& I_y(q_1) \\
	I_x(q_2) 	& I_y(q_2) \\ 
	\vdots 		& \vdots \\
	I_x(q_N)	& I_y(q_N)
\end{bmatrix} \quad 
v = 
\begin{bmatrix}
V_x \\ V_y
\end{bmatrix} \quad
b = 
\begin{bmatrix}
-I_t(q_1) \\ -I_t(q_2) \\ \vdots \\ -I_t(q_N)
\end{bmatrix}
\end{equation}

The equation set in \eqref{equ:lk.local.flow.mat} is overdetermined. The Lucas-Kanade method now uses the 
least squares method to find the minimal solution to this set. The system that then must be solved usually takes the 
form of \eqref{equ:lk.local.flow.solv}.

\begin{equation}\label{equ:lk.local.flow.solv}
v = \left(A^\top A\right)^{-1} A^\top b
\end{equation}

If equation \eqref{equ:lk.local.flow.solv} is expanded back to its original matrices and using sums, we have the system in \eqref{equ:lk.local.flow.solv.full}

\begin{equation}\label{equ:lk.local.flow.solv.full}
\begin{bmatrix}
V_x \\ V_y
\end{bmatrix} = 
\begin{bmatrix}
\sum_ i{I_x^2(q_i)} 	 & \sum_i{I_x(q_i)I_y(q_i)} \\
\sum_i{I_y(q_i)I_x(q_i)} & \sum{I_y^2(q_i)}
\end{bmatrix}^{-1}
\begin{bmatrix}
-\sum_i{I_x(q_i)I_t(q_i)} \\ -\sum_i{I_y(q_i)I_t(q_i)}
\end{bmatrix}
\end{equation}

\subsection{Horch-Schunck}
