
\chapter{Specifications}
The hardware and software used in this project was chosen to be as close as possible to the 
hardware used by Argus in their underwater vehicles. This gives us 
a scenario closer to the real hardware, and imposes the same 
restrictions on us as would be imposed in real world applications.

\section{Hardware}

\subsection{Argus hardware}
The hardware used by Argus is only mentioned as 
\begin{quote}
$1\times$ Foc/Zoom camera \\
Optional HDTV Camera 1080i \\
$1\times$ Lowlight Black \& White camera \\
\end{quote}
in \citet{argusROV}. After some mailing with Argus, they provided us 
with the specific details of this HD Camera, which is what we are interested in. 
The most interesting parts of the camera specification is shown in table \vref{tbl:fcbh11}.

\subsection{High definition video hardware}
The camera used by Argus is the Sony FCB-H11 \citet{fcbh11}. 
\begin{center}
	\begin{table}[htbp]
		\begin{tabular}{ll}
		
			\toprule
				Camera specification & Detail \\
			\midrule
				Image sensor 			& 1/3-type CMOS \\
				Pixels 					& $\approx2\times10^{6}$ Pixels \\
				Zoom 					& $12\times$ (Digital) and $10\times$ (Optical) \\
				Gain 					& Auto and manual (\SI{-3}{\deci\bel} to \SI{18}{\deci\bel}) \\
				S/N						& > 50dB \\
				Minimum illumination 	& \SI{1.2}{\lux} (F1.8 50IRE) \\
										& \SI{1.0}{\lux} (ICR ON F1.8 50IRE) \\
				Video output			& HD Analog component Y/Pb/Pr \\
										& HD Digital LVDS Y/Pb/Pr 8 bit \\
										& SD VBS \SI{1.0}{\volt_{p-p}} Negative sync Y/C \\
				Camera control interface& VISCA TTL,\\
				Operating temperature	& \SI{0}{\celsius} to \SI{45}{\celsius} \\
				Power consumption		& \SI{9}{\volt} $\pm$ \SI{3}{\volt} DC, \SI{4.8}{\watt} \\
			\bottomrule
		\end{tabular}
		\caption{Selected FCB-H11 Specifications}
		\label{tbl:fcbh11}
	\end{table}
\end{center}

As seen in \vref{tbl:fcbh11}, the camera outputs HD Digital LVDS\footnote{Low-Voltage Differential Signal} signals.
Due to the open standard used by this camera, there are quite a few different add-on cards that 
converts this LVDS signal to other, more robust signals for different applications.


\begin{figure}[htbp]
	\centering
	\includegraphics[width=0.8\linewidth]{fcbh11}
	\caption{Sony FCB-H11 High Definition Block Camera}
	\label{fig:fcb-h11}
\end{figure}

\subsection{High definition video signalling}



Many different interface boards are available for the FCB-H11 which gives different output formats from the camera and 
provides different control interfaces to the camera..
The most normal inteface cards provides HDMI, USB, GigEthernet, SDI and HD-SDI outputs in 
different configurations and combinations. A comparison of the different technologies is shown in \vref{tbl:transmission_systems}.

\begin{center}
	\begin{table}[htbp]
		\begin{tabular}{llllr}
		
			\toprule
				System & Bitrate & Distance & Protocol & Compression from FCB-H11 \\
			\midrule
				HDMI 				& \SI{10.2}{\giga bit\per\second}	& $\approx$ \SI{15}{\metre}		& TMDS 		& 0\% \\
				USB 3.0 			& \SI{5}{\giga bit\per\second}		& $\approx$ \SI{5}{\metre}		& Serial	& 0\% \\
				Gigabit Ethernet	& \SI{1}{\giga bit\per\second}		& $\approx$ \SI{220}{\metre}	& Serial	& $\approx$ 30\% \\
				SDI					& \SI{360}{\mega bit\per\second}	& $\approx$ \SI{300}{\metre}	& NRZI 		& $\approx$ 75\% \\
				HD-SDI				& \SI{1.485}{\giga bit\per\second}	& $\approx$ \SI{300}{\metre}	& NRZI		& 0\% \\
			\bottomrule
		\end{tabular}
		\caption{Different signalling systems}
		\label{tbl:transmission_systems}
	\end{table}
\end{center}


For this project, it was decided that we should 
try to get a interface that did have more capacity than the camera produces. Looking 
at the different options in \vref{tbl:transmission_systems}, and also thinking of the transmission length, HD-SDI
was chosen as the desired technology for transferring the video stream from the camera.

\subsubsection{HD-SDI}\label{sec:hdsdi}
The HD-SDI signalling standard is a improvement to the older SDI standard. Where the SDI can only transfer images
in maximum 576i format, the HD-SDI standard i capable of transferring 720p and 1080i video. Newer versions 
of the HD-SDI standard is also capable of 1080p and 4K video streams. 

HD-SDI is a professional video standard mainly used by TV stations and in other high end systems. It is 
defined and maintained by the Society of Motion Picture \& Television Engineers, SMPTE. More information 
on the SDI family can be found on the SMPTE website at \url{https://www.smpte.org/standards/}.

\begin{figure}[htbp]
	\centering
	\includegraphics[width=0.8\linewidth]{em15710}
	\caption{Intertest EM15710 iShoot-FCB-HDSDI interface}
	\label{fig:em15710}
\end{figure}

\subsubsection{Video stream grabbing}
The capture of video streams from a HD-SDI interface usually needs specifically designed hardware that 
connects to one of the internal buses in the computer to get high enough bandwidth. This is usually 
done using a PCI Express card that can support multiple HD-SDI streams simultaneously. 

This is 
however not practical, as we are going to use different computers and laptops during our testing. 
It would therefore be better to get a external capture interface that connects to the computers using a 
external computer interface. Thankfully, with the development of USB 3.0 we are able 
to support the bandwidth requirements for HD signal.

The Ultrastudio SDI from Blackmagic Design was chosen as the stream grabber. This is a industry standard 
supplier that recently developed this external capture unit. More information on this can be found at 
\url{http://www.blackmagicdesign.com/products/ultrastudiousb3/}, and the unit is shown in \vref{fig:ultrastudio_sdi}.

\begin{figure}[htbp]
	\centering
	\includegraphics[width=0.8\linewidth]{ultrastudiosdi}
	\caption{Blackmagic Ultrastudio SDI for USB 3.0}
	\label{fig:ultrastudio_sdi}
\end{figure}

The Ultrastudio SDI unit gives HD-SDI in together with HD-SDI out, HDMI out and USB 3.0. The 
extra HDMI out makes it easy to connect a external monitor while capturing to get better visual feedback.



\section{Software}

\subsection{High definition video capture}

\subsection{High definition video analysis}

\subsection{Real time constraints}

