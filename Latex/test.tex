
\chapter{Field Tests}\label{ch:field_test}

Computer Vision algorithms can only give as good result as the source videostream that it is 
being fed. After talking a bit to \gls{sintef}, some test videos were provided by them. However, as seen in figure 
\vref{fig:sintef_not_1}, the quality leaves much to be desired. The video provided had a resolution 
of $854 \times 480$ pixels with big black box padding around it. This does not only make 
the video unsuitable for us in a computer vision algorithm, but it also means that the video 
probably is upscaled quite a bit.

Due to the poor nature of the quality of the video it was decided early on that 
a field test was needed, since we were going to use equivalent hardware as to what is available on the \gls{rov}. 

\begin{figure}[htbp]
	\centering
	\includegraphics[width=0.9\textwidth]{sintef_not_video_1}
	\caption{Original video provided by \gls{sintef}}
	\label{fig:sintef_not_1}
\end{figure}

\section{SINTEF DVL Test}
As part of the cooperation with \gls{sintef} during the preliminary testing, we 
were asked to help with a \gls{dvl} test using a rig for controlling 
the movement of the net. As a favour for us helping out, we 
got to lend the rig at a later time when our hardware were ready to do a field test. 

During this test, we learned enough about the rig and operation of it that we should be able to operate 
it ourselves. The field test also gave some valuable information 
on how to get everything set up correctly, and what preparations which 
were needed to go through with our test.

\begin{figure}[htbp]
	\centering
	\includegraphics[width=0.9\textwidth]{rig1}
	\caption{Field test for \gls{sintef} using \gls{dvl}}
	\label{fig:test_dvl}
\end{figure}

\section{HD Video Test}

\begin{table}[htbp]
	\centering
	\begin{tabular}{ll}
		\toprule
			Net setup 					& Description  \\
		\midrule
			Reference					& Net in static position. No forced movement. \\
			Reference with movement		& Net lowered in front of the camera. \\
			Circular hole				& Net with a circular cut-out lowered. \\
			Vertical tear				& Net with a narrow vertical tear lowered. \\
			Horizontal tear				& Net with narrow horizontal tear lowered. \\
			L-shaped tear				& Net with horizontal L-shaped tear lowered. \\
			Growth						& Net with imitation growths lowered.\\
			Double sea-cage				& Two nets laid on top of each other.\\
		\bottomrule
	\end{tabular}
	\caption{Test setup, all at \SI{1.5}{\metre}}
	\label{tbl:test_setup}
\end{table}

\begin{figure}[htbp]
    \centering
    \subfloat[Circular Hole]{\label{fig:net_hole}{\includegraphics[width=0.3\textwidth]{net_hole}}} \hfill
    \subfloat[Vertical tear]{\label{fig:net_vertical}{\includegraphics[width=0.3\textwidth]{net_vertical}}} \hfill
    \subfloat[Double net]{\label{fig:net_double}{\includegraphics[width=0.3\textwidth]{net_double}}}
    \\
	\subfloat[L-shaped tear]{\label{fig:net_L_tear}{\includegraphics[width=0.45\textwidth]{net_l}}} \hfill
	\subfloat[Horizontal tear]{\label{fig:net_horize}{\includegraphics[width=0.45\textwidth]{net_horiz}}}
	\caption{Different net configurations with holes and tears, excluding normal single net. Images collected from \citet{sletta13}.}
	\label{fig:net_configs}
\end{figure}

\begin{figure}[htbp]
	\centering
	\includegraphics[width=0.9\textwidth]{rig2}
	\caption{Field test on a sunny day in May}
	\label{fig:test_hd}
\end{figure}

The rig was configured to mimic the approximate distance to the net 
in figure \vref{fig:sintef_not_1}. We were however not able to tilt the camera to 
an angle due to environmental constraint in a anchor chain situated below the camera.
We approximated the distance between the ROV and the net in figure \vref{fig:sintef_not_1} to be 
\SI{1.5}{\metre}. This lead to a the image in figure \vref{fig:test_hd_referanse}.

\begin{figure}[htbp]
	\centering
	\includegraphics[width=0.9\textwidth]{hd_not_all}
	\caption{View of the net from \SI{1.5}{\metre}. This is the default distance for the rig used.}
	\label{fig:test_hd_referanse}
\end{figure}

Comparing image \vref{fig:test_hd_referanse} and \vref{fig:sintef_not_1}, it seems that the 
top row of masks is approximately the same in both images. Due to the tilt of the 
camera in image \ref{fig:sintef_not_1} we get a prominent vanishing point in that image. There 
are also some growing on the net in figure \ref{fig:sintef_not_1}, which of obvious reasons does not appear 
in figure \ref{fig:test_hd_referanse}.

\begin{figure}[htbp]
	\centering
	\includegraphics[width=0.9\textwidth]{hd_not}
	\caption{View of masks in the net at \SI{1.5}{\metre}. Digitally zoomed in post processing.}
	\label{fig:test_hd_clip}
\end{figure}

The test was done in accordance with the test patterns described in figure \vref{fig:net_configs}. 