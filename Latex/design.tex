% !TeX spellcheck = nb_NO

\section{Design}
Meningen med designet av sensoren er å få laget en membran som kan legges mellom en stump og et protesehylster. Denne membranen er ment å inneholde 
sensorpunkter inspirert av teknologien brukt i berøringsfølsomme skjermer. Disse er bygget opp etter det såkalte XY-felleskapasitansprinsippet beskrevet i \citet{qtan0079}
og \citet{quantum2006}.

Kretsen som vil bli brukt i dette designet er nærmere beskrevet i appendiks \vref{appendix:mutualcapacitance.circuit}.

For å minimere målefeil og andre påvirkninger brukes et dobbelthelningsprinsipp\footnote{dual slope}for å måle ladningen som legger seg over målekondensatoren.

\subsection{Membran}\label{sec:membran}
Membranen er tenkt laget av et fleksibelt og elektrisk isolerende materiale. Dette vil føre til at trykk vil påvirke avstanden mellom målepunktene, men hvor mye er avhengig av kompressibiliteten til det materialet som blir valgt. Det er også viktig at det har god toleranse for både strekk- og skyvkrefter i alle retninger slik at membranen ikke blir ødelagt
ved lengre tids bruk.

\begin{figure}[htbp]
\centering
\begin{tikzpicture}[scale=0.75]
	\foreach \x in {1,2,...,8} {
		\draw (0,\x) -- (9,\x);
		\draw (\x,0) -- (\x,9);
		\node at (\x,0) [below] {$X_\x$};
		\node at (0,\x) [left] {$Y_\x$};
		\foreach \y in {1,2,...,8} {
			\node[draw,circle] at (\x,\y) {};
		}
	}
	
	\draw[dashed,rounded corners] (6.5,0.5) rectangle (8.5,2.5);
	\node[below right] at (8.5,0.5) {Se figur \vref{fig:3d_cell}};
\end{tikzpicture}
\caption{Skjema av målepunkter}
\label{fig:sampl_point}
\end{figure}

Hvis vi ser på hvordan den stiplede boksen i figur \vref{fig:sampl_point} ser ut i 3 dimensjoner blir det lettere å referere figuren til det vi ønsker å måle som 
ble vist i kapittel \vref{sec:measure_cap}. Dette utdraget er vist i figur \vref{fig:3d_cell}.

\begin{figure}[htbp]
\centering
\begin{tikzpicture}[scale=1.75]
	%%% Edit the following coordinate to change the shape of your
	%%% cuboid
      
	%% Vanishing points for perspective handling
	\coordinate (P1) at (-7cm,1.5cm); % left vanishing point (To pick)
	\coordinate (P2) at (8cm,1.5cm); % right vanishing point (To pick)

	%% (A1) and (A2) defines the 2 central points of the cuboid
	\coordinate (A1) at (0em,0cm); % central top point (To pick)
	\coordinate (A2) at (0em,-2cm); % central bottom point (To pick)

	%% (A3) to (A8) are computed given a unique parameter (or 2) .8
	% You can vary .8 from 0 to 1 to change perspective on left side
	\coordinate (A3) at ($(P1)!.8!(A2)$); % To pick for perspective 
	\coordinate (A4) at ($(P1)!.8!(A1)$);

	% You can vary .8 from 0 to 1 to change perspective on right side
	\coordinate (A7) at ($(P2)!.7!(A2)$);
	\coordinate (A8) at ($(P2)!.7!(A1)$);

	%% Automatically compute the last 2 points with intersections
	\coordinate (A5) at
	  (intersection cs: first line={(A8) -- (P1)},
			    second line={(A4) -- (P2)});
	\coordinate (A6) at
	  (intersection cs: first line={(A7) -- (P1)}, 
			    second line={(A3) -- (P2)});

	\foreach \i in {1,2,...,8}
	{
	  \draw[fill=black] (A\i) circle (0.15em)
	    node[above right] {};
	}
	
	%\draw (A1) -- (A8);
	\draw[thick] (A3) -- (A6);
	%\draw (A4) -- (A5);
	\draw[thick] (A2) -- (A7);
	\draw[thick] (A1) -- (A4);
	\draw[thick] (A8) -- (A5);
	%\draw (A6) -- (A7);
	%\draw (A2) -- (A3);
	\draw[dashed,thin] (A4) -- (A3);
	\draw[dashed,thin] (A2) -- (A1);
	\draw[dashed,thin] (A8) -- (A7);
	\draw[dashed,thin] (A5) -- (A6);
	
	\node[left] at ($(A4)!.5!(A3)$){$d(t)$};
	\node[left] at ($(A2)!.3!(A1)$){$d(t)$};
	\node[left] at ($(A8)!.5!(A7)$){$d(t)$};
	\node[left] at ($(A5)!.7!(A6)$){$d(t)$};
	
	\node[below] at ($(A3)!.3!(A6)$) {$X_7$};
	\node[below] at ($(A2)!.5!(A7)$) {$X_8$};
	\node[above] at ($(A8)!.5!(A5)$) {$Y_2$};
	\node[above] at ($(A4)!.5!(A1)$) {$Y_1$};
\end{tikzpicture}
\caption{4 målepunkt mellom X- og Y-planet}
\label{fig:3d_cell}
\end{figure}

I figur \vref{fig:3d_cell} vil \(d(t)\) kunne være forskjellig fra punkt til punkt ettersom hvordan trykket 
er fordelt utover membranen. Det er verdt å merke seg med at membranen er tøyelig og dermed innehar viskøse egenskaper. 
Dette vil fører til at ett trykkpunkt vil kunne gi utslag på flere målepunkt sirkulært rundt trykkpunktet. Signalet
vil da få en form med et maksima ved trykkpunktet, og mindre utslag fordelt rundt dette. Grafisk vil dette 
ser ut som figur \vref{fig:gaussian}.

\begin{figure}[htbp]
	\centering
	\includegraphics[width=\textwidth]{img/gaussian}
	\caption{Gaussisk funksjon}
	\label{fig:gaussian}
\end{figure}

\subsubsection{Bygging}\label{sec:bygging}
Selve membranen ble bygget av en temperaturbestandig silikon, da denne er lett tilgjengelig, billig og mer tøyelig enn andre 
silikonprodukter. Først ble ett lag med silikon spredt utover en plate med slippbelegg slik at silikonen ikke skulle feste seg 
til platen under herding. Når dette laget hadde begynt å stivne, ble alle elektrodene plassert på en rad, og ett nytt lag med silikon lagt
på toppen med en ny plate med slippbelegg på toppen.

Da denne raden hadde stivnet, ble den delt på midten og den ene halvparten festet på toppen av den første 
snudd \SI{90}{\degree} i forhold til aksen til bunnen, slik at en konfigurasjon som vist i figur \vref{fig:sampl_point} og figur \vref{fig:membran} ble oppnådd.

Ferdig resultat er vist i figur \vref{fig:membran} og testoppsettet er vist i figur \vref{fig:oppsett}.

\begin{figure}[htbp]
	\centering
	\includegraphics[width=\textwidth]{img/membran.jpg}
	\caption{Membran i testoppsett}
	\label{fig:membran}
\end{figure}

\begin{figure}[htbp]
	\centering
	\includegraphics[width=\textwidth]{img/oppsett.jpg}
	\caption{Testoppsett}
	\label{fig:oppsett}
\end{figure}

\subsection{Egenskaper og problemer}
Ettersom designet beskrevet i \vref{sec:bygging} bruker silikon som byggemateriale, er bevegelsen til membranen begrenset av elastisiteten til silikonet sammen med 
de skjærkreftene som kommer fra elektrodene som er støpt fast i silikonet. Elektrodene kan ikke bevege seg i silikonet, og har derfor en gitt utgangsposisjon som 
membranen vender tilbake til.

Som et resultat av oppbygningen, så er membranen bøyelig i alle retninger. Elektrodene som ble brukt var litt for tykke, noe som førte til at 
membranen ble relativt stiv.
Målinger som ser på kryssnakk bli da vanskelig å få til, da det krever å holde membranen i spenn samtidig som det legges trykk på den.
