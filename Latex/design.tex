
\chapter{Field Tests}\label{ch:field_test}

Computer Vision algorithms can only give as good result as the source videostream that it is 
being fed. After talking a bit to SINTEF, some test videos were provided by them. However, as seen in figure 
\vref{fig:sintef_not_1}, the quality leaves much to be desired. The video provided had a resolution 
of $854 \times 480$ pixels with big black box padding around it. This does not only make 
the video unsuitable for us in a computer vision algorithm, but it also means that the video 
probably is upscaled quite a bit.

Due to the poor nature of the quality of the video it was decided early on that 
a field test was needed, since we were going to use equivalent hardware as to what is available on the ROV. 

\begin{figure}[htbp]
	\centering
	\includegraphics[width=0.9\textwidth]{sintef_not_video_1}
	\caption{Original video provided by SINTEF}
	\label{fig:sintef_not_1}
\end{figure}

\section{SINTEF DVL Test}
This lead to us helping SINTEF out with a doppler velocity log test using a rig for controlling 
the movement of the \todo{not som i garn?}. As a favour for us helping out, we 
got to lend the rig at a later time when our hardware were ready to do a field test. 

During this test, we learned enough ab the rig and operation of it that we should be able to operate 
it ourselves.

\begin{figure}[htbp]
	\centering
	\includegraphics[width=0.9\textwidth]{rig1}
	\caption{Field test for SINTEF using DVL}
	\label{fig:test_dvl}
\end{figure}

\section{HD Video Test}

\begin{figure}[htbp]
	\centering
	\includegraphics[width=0.9\textwidth]{rig2}
	\caption{Field test}
	\label{fig:test_hd}
\end{figure}

The rig was configured to mimic the approximate distance to the net 
in figure \vref{fig:sintef_not_1}. We were however not able to tilt the camera to 
an angle due to environmental constraint in a anchor chain situated below the camera.
We approximated the distance between the ROV and the net in figure \vref{fig:sintef_not_1} to be 
\SI{1.5}{\metre}. This lead to a the image in \vref{fig:test_hd_referanse}

\begin{figure}[htbp]
	\centering
	\includegraphics[width=0.9\textwidth]{hd_not_all}
	\caption{View from \SI{1.5}{\metre}}
	\label{fig:test_hd_referanse}
\end{figure}

Comparing image \vref{fig:test_hd_referanse} and \vref{fig:sintef_not_1}, it seems that the 
top row of masks is approximately the same in both images. Due to the tilt of the 
camera in image \ref{fig:sintef_not_1} we get a prominent vanishing point in that image. There 
are also some growing on the net in \ref{fig:sintef_not_1}, which of obvious reasons does not appear 
in \ref{fig:test_hd_referanse}.

\begin{figure}[htbp]
	\centering
	\includegraphics[width=0.9\textwidth]{hd_not}
	\caption{Field test}
	\label{fig:test_hd_clip}
\end{figure}
