
\chapter{Introduction}
During the last four decades, the demand for fish 
has been steadily increasing. The current level 
of demand supersedes the amount of fish that 
can be fished and keep the different stocks of 
fish at a sustainable level. This has lead to 
overfishing, eliminating for instance the 
fish population on the Grand Banks off the east 
coast of America. Many other fish 
populations are also either gone or in fast 
diminish. Some scientists are even predicting 
that with the current overfishing, the fish industry will 
see a global collapse during this century \citet{worm06}.

To combat the overfishing of natural fish stocks are to increase 
the production of fish through farming. This way of ensuring 
sustainability for both stocks and people has been 
used for millenias on land, and is now being rapidly 
deployed on sea. According to \citet{fao06}, the amount of 
fish that were farmed globally in 2005 was 48.1 million tons, out of 
a total 141.3 million tons of fish that were produced. This 
means that about a third of all fish sold were farmed in 2005, with 
steadily increasing numbers.




\section{Motivation}


\section{Previous Work}


\section{Outline}

